\chapter{Solidity Smart Contract}
\label{sec:solidity_contract}

The solidity contract is in \href{https://github.com/frenzis01/mastermind/blob/400da112942becdf070796144fe0db0022c481b6/contracts/Mastermind.sol#L228}{\texttt{contracts/Mastermind.sol}}. 
The variables which could be statically and arbitrarily defined are set as constants at the beginning of the contract, to allow ease of tuning of such parameters.

Everything revolves around a \texttt{Game} struct which holds the state of a game.
\ul{Alterations to such state are logged in the form of \textit{events}}, each one associated to a function.

% "valid" o "sound"
\lstinline{modifier}s are used to ensure that at the time of function's invocation \ul{its preconditions are satisfied, ensuring that the state of the contract is always valid}.
Functions marked with \lstinline{view} \ul{do \textit{not} modify the state of the contract}, and there is \textbf{no fee} to execute them, unless they are executed inside a function which do alters the state: in such case, they contribute to the function invocation's fee.
\lstinline{external view} functions are used only by the frontend or by Mocha tests (\texttt{.js} files under \texttt{tests} folder) to get information on the games, \lstinline{internal view} ones only by other functions in the contract, while \lstinline{public view} match both use cases.\\
The only \lstinline{payable} functions are \texttt{createGame} and \texttt{joinGame}.


\section{Core functions and Choices}
\label{sec:core-functions}

\subsection{Creating and Joining games}
\label{sec:create-join-games}
The ``agreement" on a stake to play on is handled in the frontend, and basically consists in choosing the game with the stake closest to desired one. 
For what concerns the contract, \texttt{createGame()} specifies a stake, and a joiner simply invokes \texttt{joinGame()} sending such stake.
The two functions (mostly the former) initialize most of the \texttt{Game} state, but not all: some informations are set by \texttt{startTurn()} ---discussed in \ref{sec:startTurn}--- , since they need to be initialized not only at the beginning of the game, but at each turn. 


\subsection{Feedbacks and the end of a turn}
{The logic behind \texttt{makeGuess()} and \texttt{provideFeedback()} is rather simple, but the latter must handle two special cases:\ns
\begin{enumerate}
    \item The feedback provided indicates that $CC$\footnote{Number of colors in the correct positions} is equal to the length of the code (6 in the screenshots provided earlier), indicating that the \ul{breaker has \textbf{guessed} the code}.\\
    At this point the turn should end, having the \textgreen{breaker} guessed the secret code.
    \item The \ul{breaker has \textbf{exhausted} their available guesses}, so the turn should end, with the \textmauve{maker} being awarded of extra points, since the \textgreen{breaker} could not guess its code.
\end{enumerate}}

\begin{lstlisting}
function provideFeedback(
    uint256 _gameId,
    uint256 numCorrectPositions,
    uint256 numCorrectColors    // correct colors but in wrong positions!
) external checkGameValidity(_gameId) turnNotEnded(_gameId) {
    Game storage game = games[_gameId];
    // Ensure feedback is valid
    require(
            numCorrectColors + numCorrectPositions <= game.codeLength,
        "Invalid feedback"
    );
    // Ensure player has not already provided feedback in the current turn
    require(
        isMakerTurn(_gameId),
        "Code maker has already provided feedback");
    game.feedbacks[game.currentTurn].push([numCorrectPositions, numCorrectColors]);
    game.accusedAFK[msg.sender] = 0; // Reset the AFK accusation if present
    emit Feedback(_gameId, numCorrectPositions, numCorrectColors);
    if (numCorrectPositions == game.codeLength) {
        endTurn(_gameId,true);
        return;
    }
    if (game.guesses[game.currentTurn].length == game.maxGuesses) {
        endTurn(_gameId, false);
    }
}
\end{lstlisting}

Note that the \ul{smart contract is \textbf{not} able to check that the feedback is consistent with the secret code} chosen by the \textmauve{Maker}. The secret code is unknown to the smart contract until the end of a turn when it gets published using \texttt{publishCodeSecret()}, but even then, the contract does not check that all received feedbacks are consistent with the given Secret Code: doing so would invalidate the requested \textit{Dispute} feature.

\subsection{Secret - Commit and reveal}
\label{sec:secret_hash}
About \texttt{publishCodeSecret()} and \texttt{submitCodeHash()}, to avoid involving the contract in any way in the hash computation, the contract simply receives a \lstinline{bytes32 hash} value right after the turn has started, without ever knowing the values which produced it, but actually expecting it had been computing using a precise algorithm and data structure; we'll cover this aspect in a moment.  

To inhibit the risk of a brute force attack (or lists of publicly known mappings $value\rightarrow hash$), we chose to prepend a \textit{salt}\footnote{This is called \lstinline{seed} in the code} to the secret code before hashing. 
In the solidity contract, \texttt{publishCodeSecret(gameId, secret, seed)} computes the hash by invoking the following function, and checks that the result is equal to the hash submitted at the start of the turn.
\begin{lstlisting}
function hashArrayOfIntegers(uint256[] memory intArray, bytes memory seed) public pure returns (bytes32) {
    // Encode the array using abi.encodePacked
    bytes memory encodedArray = abi.encodePacked(intArray);
    // Combine the seed and the encoded array
    bytes memory combined = abi.encodePacked(seed, encodedArray);
    // Hash the encoded array using keccak256
    bytes32 h = keccak256(combined);
    return h;
}
\end{lstlisting}

The key point is that \ul{on the frontend side the hashing function must perfectly mimick the behavior of this one}, in order to produce the same hash (given the same inputs) and for the comparison in \texttt{publishSecretCode()} to succeed.

Below is listed the frontend code to compute such hash, which required some tuning and even though it looks more complicated than solidity one, it does the exact same thing.
The randomness of the salt relies on\\ \href{https://developer.mozilla.org/en-US/docs/Web/API/Crypto/getRandomValues}{\texttt{window.crypto.getRandomValues()}}, which is reported as \textit{``suitable for cryptographic purposes"}
\begin{lstlisting}
// Generate a random 64-character string salt
generateRandomSeed() {
  const array = new Uint8Array(32);
  window.crypto.getRandomValues(array);
  return array; // 32 bytes * 2 hex chars per byte = 64 hex chars
}
// Hash the salt prepended to the serialized array
computeHash(intArray, seed) {
  // Serialize the array as it would be in Solidity
  const abiCoder = ethers.AbiCoder.defaultAbiCoder();
  const types = new Array(intArray.length).fill('uint256');
  const serializedArray = abiCoder.encode(types, intArray);
  // Concatenate the seed and the serialized array
  const combined = ethers.concat([seed, serializedArray]);
  // Compute the hash
  return ethers.keccak256(combined);
}
\end{lstlisting}

\subsection{AFK - Away from Keyboard accusation}

The contract allows to accuse the opponent of being AFK, and they do not perform actions on the game for \ul{$B$ blocks}, the accuser may win the entire stake.\\
In solidity there is no setting of timers or triggers, so the act of accusing and claiming the stake is split in two functions:
\begin{enumerate}
    \item \texttt{accuseAFK()}: Which notes in \texttt{game.accused[accused]} the current block number.\\
    Such accusation note (if present) is automatically reset every time the accused player invokes a function associated to an in-game action; \texttt{createGame()}, \texttt{joinGame()} and \texttt{accuseAFK()} itself are not considered in this set).
    It was chosen for \texttt{accuseAFK()} not to reset the \textit{accuser} accusation note, since this may lead to the two players to indefinitely accuse theirselves without actually playing the game.\\
    In order for the function to be invoked \ul{it must be the turn of the accused};
    this is crucial for the correct functioning of the AFK accusation.
    \item \texttt{verifyAFKAccusation()}: Which checks whether the accused had been AFK for enough time, and if so, ends the game, transferring the game stake to the accuser.
\end{enumerate}

\subsection{Disputing}
When the \textgreen{breaker} invokes the \lstinline{disputeFeedback()} they must attach the \textit{set} of feedback IDs they are disputing.\\
Such function can be invoked only after the turn has ended, and after the code secret has been published.

The contract, having now the secret code in clear text, may compute the correct feedbacks for the given guesses\footnote{The ones corresponding to the feedbacks indicated by the \textgreen{Breaker}} and compare them with the ones stored in the game state (\texttt{game.feedbacks[currentTurn]}).

It is important to note that \textit{\ul{all}} the feedbacks indicated by the \textgreen{breaker} must result invalid for the \textgreen{Breaker} to win, if any of them is correct the \textmauve{Maker} wins instead;
in any case, the game ends.

\subsection{Starting Turns}
\label{sec:startTurn}

There are the two main design choices concerning \lstinline{startTurn()}, which however affect how the whole game proceeds.
\nl

Even if it seems counterintuitive ---since it's the \textmauve{maker} who has to first publish the secret hash--- , \ul{it's the}\textgreen{breaker} \ul{who has to start a turn}, the \textmauve{maker} is not allowed to do so.
We chose this way mostly due to the logic of the end turn phase and of the dispute.\\
Consider this scenario: the turn has ended, and the \textmauve{maker} has revealed the secret code, but the next turn has \textit{not} started yet.
At this point the \textgreen{breaker} has a fixed period of time to review the feedbacks provided by the \textmauve{Maker} and decide whether it's the case to \textbf{dispute} some of them. 
It makes sense that is the \textgreen{Breaker} to say at some point ``Okay, nothing to dispute, we can move onto the next turn", rather than the \textmauve{Maker} to ask the contract whether the time for disputing has elapsed or the \textgreen{Breaker} is ready to start the new turn;
the latter option would imply also an additional transaction performed by the \textgreen{Breaker} to state ``Okay, nothing to dispute".

\lstinline{startTurn()} also handles the ``standard" way of ending a game: when the current turn reaches the maximum, an invocation to \lstinline{startTurn()} results in the normal end of the game, comparison of points, and stake transfer to the winner. A detailed \textbf{interaction schema} is provided in Appendix \ref{fig:mastermind_schema}.
{\ns\note{In case of a tie, the stake goes to the game \textit{creator}}}

\section{Testing}

Since the contract's first core functions were implemented, further developments were accompanied by Mocha test suites.
Such test suites are in \href{https://github.com/frenzis01/mastermind/tree/6b5accf8a56b1935e354d0887ca133f5b1482c09/test}{test} folder.
The absence of a GUI and user inputs allowed to quickly write and run (unit) tests.
Tests may be ran by executing:
\begin{lstlisting}[language=bash]
npx hardhat test
\end{lstlisting}

{There are three \texttt{.js} test files:
\begin{enumerate}
    \item \texttt{Mastermind.js} tests all core functions of the contract, and handle some scenarios. 
    Even if Mocha is fully exploited with independent unit tests, in this case units depend on the previous state, resulting in a more linear testing. 
    \item \texttt{Looped.js} simply executes the same \texttt{Mastermind.js} code but multiple times, it was used in late development to obtain more precise gas reports (described in the following section \ref{sec:gas_report}).
    \item \texttt{Parallel.js} runs multiple games in a row.
\end{enumerate}}